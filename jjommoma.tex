\documentclass{thiguka}
\usepackage{bookmark}
\usepackage{multirow}
\usepackage{tabularx}
\usepackage{lmodern}
\usepackage{multicol}
\usepackage[margin=4em]{geometry}

\usepackage{gb4e}
\setlength{\glossglue}{5pt plus 2pt minus 1pt}

\usepackage{array}
\usepackage{tipa}
\usepackage{textcomp}

\usepackage{fontspec}
\setmainfont{Gentium Plus}

\usepackage{booktabs}
\usepackage{tabularx}

\usepackage{iftex}
\iftutex
  \DeclareFontFamilySubstitution{T3}{\rmdefault}{cmr}
\fi

\begin{document}

Ambagame3 Translation

From: jjommoma

To: Lemuria

June 16, 2024

\section{Translation}
\subsection{Thiguka}
Fipelupahelah falusaya geguelah geylepahelah.
Isa geylepafaysa leylepahsa lasa “Sutepafay-rukahba.”\\
Sutepafay-rukahbakala, pelupah sutepahelah suguiba sutepafayelah ibilayelah.
Geylepafaypa sifarlarikigalpahsa lasa rakuka sigutu sifarabukakuspafay sutepafayelah ibilaktay.
Pelupah baseba itala sutetayelah.
Geylepahsa sakara kula isa giledpahsa pafay sutetaypa.

\subsection{Lemuria's translation}
Children like many games.
One of them is "Stone Fighting", where players must throw stones at other stacks of stones and topple them.
Players can attempt to catch the stones as they are hurled at the stacks.
The game will end once one group has no remaining stones.

\subsection{Gloss}

\begin{exe}
\ex{} \gll{}Fi-pelu-pah-elah falusaya ge~gu-elah geyle-pah-elah.\\
            DIM-person-NOM-PL like AGR~ADJ-many game-NOM-PL.\\
      \glt{}Children like a lot of games.
\end{exe}

\begin{exe}
\ex{} \gll{}Isa geylepafaysa leylepahsa lasa "Sute-pafay---rukahba."\\
            one game-GEN-SG name-NOM-SG COP.1SG.PRS stone-GEN-fight.LOAN.V\\
      \glt{}One of these games is named "Stone Fighting".
\end{exe}

\begin{exe}
\ex{} \gll{}Sute-pafay---rukahba-kala, pelu-pah sute-pah-elah su~gu-iba sute-pafay-elah ibi-lay-elah.\\
            stone-GEN-fight.LOAN-LOC, person-NOM stone-NOM-PL AGR~ADJ-other stone-GEN-PL stack.LOAN-DAT-PL\\
      \glt{}In Stone Fighting, people throw stones at other stacks of stones.
\end{exe}

\begin{exe}
\ex{} \gll{}Geyle-pafay-pa sifarlarikigal-pah-sa lasa rakuka su~gu-tu sifarabukakus-pafay sutepafayelah ibilak-tay.\\
            game-GEN-ZERO objective.LOAN-NOM-PL COP.1SG.PRS topple AGR~ADJ-all opponent-GEN stone-GEN-PL stack.LOAN-ACC\\
      \glt{}The game's objective is to topple the opponent's every stack of stones.
\end{exe}

\begin{exe}
\ex{} \gll{}Pelu-pah baseba itala sute-tay-elah.\\
            person-NOM can.LOAN catch.LOAN stone-ACC-PL\\
      \glt{}People can catch stones.
\end{exe}

\begin{exe}
\ex{} \gll{}Geyle-pah-sa sakara kula isa giled-pah-sa pafay sute-tay-pa.\\
            game-NOM-SG stop if one group-NOM-SG has stone-ACC-ZERO\\
      \glt{}The game stops if one group has no stones left.
\end{exe}

\section{Commentary}
\begin{enumerate}
    \item The long dash --- indicates dashes in original source text that are supposed to be there.
    \item *''geylepahelah'' should be marked as ''geyletayelah'', since it is accusative; being the object.
    \item I interpreted ''ibilayelah'' as the dative plural form of ''ibilak'' ("stack, pile").
    \item I am confused by "Geyle-pafay-pa"; it means "no game", but it may also be a misspelling of "pah".
    \item Using the 1SG.PRS copula ''lasa'' is incorrect here; you are not the game, you are talking about the game in the third person perspective. I would have used ''gulaya'' here.
\end{enumerate}

\newpage

\section{New coins}
\begin{enumerate}
    \item rukahba, ``to fight'', from ruqawūd
    \item ibilak ``tack, pile'', from ibilak
    \item sifarlarikigal ``oal, objective'', from şifarnarikigan
    \item rakuka, ``to topple'', from raqykad
    \item sifarabukakus, ``opponent'', from şifarabukakyz
    \item baseba, ``to be able to; can'', from majebad
    \item itala, ``to catch'', from yetanud
    \item tu, ``all'', from tu
\end{enumerate}

\end{document}

\documentclass{thiguka}
\usepackage{bookmark}
\usepackage{multirow}
\usepackage{tabularx}
\usepackage{lmodern}
\usepackage{multicol}
\usepackage[margin=4em]{geometry}

\usepackage{gb4e}
\setlength{\glossglue}{5pt plus 2pt minus 1pt}

\usepackage{array}
\usepackage{tipa}
\usepackage{textcomp}

\usepackage{fontspec}
\setmainfont{Gentium Plus}

\usepackage{booktabs}
\usepackage{tabularx}

\usepackage{iftex}
\iftutex
  \DeclareFontFamilySubstitution{T3}{\rmdefault}{cmr}
\fi

\begin{document}

Ambagame3 Translation

From: f0rm0r

To: Lemuria

June 18, 2024

\section{Translation}
\subsection{Thiguka}
Thakapafay giledkala -- pothupafay leylepah Pafasithsiputh-pah gulaya -- rifaripah rigurigatari gulaya.\\
Patalay, tarifaripah peguriolota pelupah lores.
Tarifagetapah kuyrikala uskesikala leyborosa; tagulikageta tarifaripah salahfaskalaelah silifatapafayelah seregkala ofefisa.
Rafupafay lahgurigatari lahrithipafay leylepah uksoth-pah gulaya.
Tarifagetapah li sipithitayelah li tarifagetasa, dasili sipithipafayelah tarifageta likolagapah gulaya.

\subsection{Lemuria's translation}
In my group, "Pafasithsiputh", musicians are important.
We are honorable musical people.
Many of our cities have temple choirs with vocalists and instrumentalists.
Their most valued instruments are the "Uksoth".
Vocalists cannot sing words because the words are impossible to sing.

\subsection{Gloss}
\begin{exe}
    \ex{} \gll{}Thaka-pafay giled-kala -- pothu-pafay leyle-pah ''Pafasithsiputh''-pah gulaya -- rifari-pah ri~gu-rigatari gulaya.\\
               1SG-GEN guild-LOC {} 3SG-GEN name-NOM ?.PR-NOM COP.3SG.PRS {} musician-NOM AGR~ADJ-important COP.3SG.PRS\\
          \glt{}In my group, "Pafasithsiputh", musicians are important.
\end{exe}

"rigatari" is used to mean "important", even though the correct word is "igalari". The form "rigatari" arose from Lemuria misremembering his lexicon and is now considered a proscribed deprecated synonym of "igalari".

\begin{exe}
    \ex{} \gll{}Pata-lay, tarifari-pah pe~gu-ri-olota pelu-pah lores.\\
                1PL.EXCL-DAT, musician-NOM AGR~ADJ-INT-honor person-NOM COP.3SG.PRS\\
          \glt{}We are honorable musical people.
\end{exe}

\begin{exe}
    \ex{} \gll{}Tarifageta-pah kuyri-kala uskesi-kala leyboro-sa; ta~gu-li-kageta tarifari-pah salahfas-kala-elah silifata-pafay-elah sereg-kala ofefi-sa.\\
                vocalist-NOM choir-LOC temple-LOC work-PRS AGR~ADJ-NEG-mouth musician-NOM group-LOC-PL city-GEN-PL between-LOC move-PRS\\
          \glt{}Many of our cities have temple choirs with vocalists and instrumentalists.
\end{exe}

\begin{exe}
    \ex{} \gll{}Rafu-pafay lah~gu-rigatari lahrithi-pafay leyle-pah \textit{uksoth}-pah gulaya.\\
                3SG-GEN AGR~ADJ-important music.instrument-GEN name-NOM uksoth.PR-NOM COP.3SG.PRS\\
          \glt{}Their most valued instruments are the "Uksoth".
\end{exe}

\begin{exe}
    \ex{} \gll{}Tarifageta-pah li sipithi-tay-elah li tarifageta-sa, dasili sipithi-pafay-elah tarifageta li-kolaga-pah gulaya.\\
                vocalist-NOM NEG word-ACC-PL NEG sing.V-PRS because word-GEN-PL sing NEG-allow-NOM COP.3SG.PRS\\
          \glt{}Vocalists cannot sing words because the words are impossible to sing.
\end{exe}

\newpage

\section{Commentary}
\begin{enumerate}
    \item The preposition "ifil" ("on, in, at") indicates a noun as being in a location, though it is not explicitly required.
    \item Copulas are in the wrong places; they should be between subject and predicate (\textit{Kakepah gulaya gulid}, not \textit{*Kakepah gulid gulaya}.)
\end{enumerate}

According to the original document: "Words in italics are nativized Žskđ words."

\section{New coins}

\begin{enumerate}
\item uskesi -- n. temple \\ ← from Žskđ ’zđ ‘deity’ + Thiguka kesi ‘building’
\item sipithi -- n. word \\ ← from Thiguka thigusipik thil ‘speaking thing’
\end{enumerate}

\end{document}

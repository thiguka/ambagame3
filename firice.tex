\documentclass{thiguka}
\usepackage{bookmark}
\usepackage{multirow}
\usepackage{tabularx}
\usepackage{lmodern}
\usepackage{multicol}
\usepackage[margin=4em]{geometry}

\usepackage{gb4e}
\setlength{\glossglue}{5pt plus 2pt minus 1pt}

\usepackage{array}
\usepackage{tipa}
\usepackage{textcomp}

\usepackage{fontspec}
\setmainfont{Gentium Plus}

\usepackage{booktabs}
\usepackage{tabularx}

\usepackage{iftex}
\iftutex
  \DeclareFontFamilySubstitution{T3}{\rmdefault}{cmr}
\fi

\begin{document}

Ambagame3 Translation

From: firice

To: Lemuria

June 14, 2024

(converted to \LaTeX{} June 16, 2024)

\section{Translation}
\subsection{Thiguka}

Kabusta, thakapafay leylepah lasa Fairistay. Thakapah sakhánjkuntha kaketay isa isi
Tabaksipafay dzinjrýnglayelah. Patapah palahkelata geyletay pothupafay leylepah lasa
Thasi Dsilrilatay, papatirayta "dzinjrýngpah sadasayta." Thasi Dsilrilapah pagasarisa 
hytsa sagysa sadzinjrása iba peluelah. Kula kalepah palifasathasa iba peluelah kalese 
ligawralay, kalepah panjínzasa.
                            
\subsection{Lemuria's translation}
Hello, my name is Firice. I will be teaching you one of Tabaxym's hunts. We made a
game about jumping and hunting other people. If you remove other people from the
game. Its name is Thasi Dsilrila, translated to "deadly hunt". Thasi Dsilrila is a
ground, you win.


\subsection{Gloss}

\begin{exe}
\ex{} \gll{}Kabusta, thaka-pafay leyle-pah lasa        Fairis-tay.\\
            hello,   1SG-GEN     name-NOM  COP.1SG.PRS Firice.PR-ACC\\
      \glt{}Hello, my name is Firice.
\end{exe}

\begin{exe}
\ex{} \gll{}Thaka-pah sa-khánjkun-tha kaketay isa isi Tabaksi-pafay  dzinjrýng-lay-elah.\\
            1SG-NOM   IPFV-teach-FUT  2SG-ACC one of  Tabaxym.PR-GEN hunt-DAT-PL\\
      \glt{}I will be teaching you one of Tabaxym's hunts.
\end{exe}

\begin{exe}
\ex{} \gll{}Patapah  pa-lahkela-ta geyle-tay pothu-pafay leyle-pah lasa        Thasi Dsilrila-tay,\\
            1PL.EXCL PFV-make-PST  game-ACC  3SG-GEN     name-NOM  COP.1SG.PRS Thasi Dsilrila-ACC\\
      \glt{}We made a game. Its name is Thasi Dsilrila,
\end{exe}

\begin{exe}
\ex{} \gll{}pa-patiray-ta     "dzinjrýng-pah sa-dasay-ta."\\
            PFV-translate-PST  hunt-NOM      IPFV-die-PST\\
      \glt{}translated to "deadly hunt".
\end{exe}

\begin{exe}
\ex{} \gll{}Thasi Dsilrila-pah pa-gasari-sa   hytsa sa-gy-sa      sa-dzinjrá-sa   iba   peluelah.\\
            Thasi Dsilrila-NOM PFV-action-PRS with  IPFV-jump-PRS IPFV-hunt.V-PRS other people-ACC\\
      \glt{}Thasi Dsilrila is a game about jumping and hunting other people.
\end{exe}

\begin{exe}
\ex{} \gll{}Kula kake-pah pa-lifasatha-sa iba   peluelah  kalese li-gawralay, kake-pah pa-njínza-sa.\\
            if   2SG-NOM  PFV-remove-PRS  other people-PL from   NEG-ground   2SG-NOM  PFV-win-PRS\\
      \glt{}If you remove other people from the ground, you win.
\end{exe}

\section{Commentary}
\begin{enumerate}
    \item So far this is one of the easiest ones. It was done in about 30-40 minutes according to the time tracker in my productivity app.
\end{enumerate}

\newpage

\section{New coins}
\begin{enumerate}
    \item silra, ``to hunt'', from dzinjrá
    \item silrili, ``hunt'', from dzinjrýng
    \item gi, ``to jump'', from gy
    \item itsa, ``with'', from hytsa
    \item kalkulu, ``to teach'', from khánjkun
    \item lilsa, ``to win'', from njínza
\end{enumerate}

khánjkun / kalkulu has a native Thiguka word:

\begin{enumerate}
    \item kalath, ``to teach, to give knowledge''
    \item fahkalath, ``teacher''
\end{enumerate}
  
The lexicon mistakenly has "kalath" as "teacher".

\end{document}

\documentclass{thiguka}
\usepackage{bookmark}
\usepackage{multirow}
\usepackage{tabularx}
\usepackage{lmodern}
\usepackage{multicol}
\usepackage[margin=4em]{geometry}

\usepackage{gb4e}
\setlength{\glossglue}{5pt plus 2pt minus 1pt}

\usepackage{array}
\usepackage{tipa}
\usepackage{textcomp}

\usepackage{fontspec}
\setmainfont{Gentium Plus}

\usepackage{booktabs}
\usepackage{tabularx}

\usepackage{iftex}
\iftutex
  \DeclareFontFamilySubstitution{T3}{\rmdefault}{cmr}
\fi

\begin{document}

Ambagame3 Translation

From: camelCaseCo

To: Lemuria

June 16, 2024

\section{Translation}
\subsection{Thiguka}
Capitalization preserved as-is from the original text.

\begin{quotation}
saguelah geyle saparakithielah basko pelupahsa kalesure.
geguisa geylekithisa kakaslesure, kasapafay leylepah kas ifas.
fasahri apalkala pegukata peluelahpafay kalahkela diledgulatay, ka kadadiledkalasa fathas litalokala katataysa.
kadadiledpafaysa paslapahsa esahla, ka paslapahgula las laki, ka las susokala las kiksesekala.
kadapaslakalagula ifigilepelupahelah kufa sugutalo sutepahelah lores gawralaykala asila paslakala.
geylepahsa kaberethata, ka ifigilepelupahelah kalakad paguiba paslakalasa, ka kasakaki sutetayelah.
\end{quotation}


\subsection{Lemuria's translation}
The Basko have a great variety of games and fun activities. One of these games is Ifas.
In Ifas, ten people from four places merge into groups of two.
Each group sees ten places, not three.
Two of these places are big and have forests and fields.
Since the first games of Ifas, people have walked to places far and wide in search of stones.

\subsubsection{Literal}
Many games stopped Basko people's fun.
One game was used for continued leisure; this game was Ifas.
Ten people from four locations split into two groups.
Each guild saw not three, but ten.
Each guild's place existed, so two places were big, and forests were fields.
In each and every place, a member would throw three stones onto that ground.
The game was born, and people would walk to places to look for stones.


\subsection{Gloss}
Glossing abbreviations:
\begin{enumerate}
    \item GDN --- General discourse marker
    \item TFX --- Typo was fixed (the original text misspelt giled as *diled)
\end{enumerate}

\begin{exe}
    \ex{} \gll{}sa\~{}gu-elah geyle sapara-kithi-elah basko pelu-pah-sa ka-lesure.\\
                AGR\~{}ADJ-PL game stop-INS-PL Basko.PR person-NOM-SG GDN-fun.activity\\
          \glt{}many games stopped Basko people's fun
\end{exe}

Why is "person" marked here in the nominative?
Why is there no accusative?
I am choosing to interpret this as ``many games stopped Basko people''.

As for lesure, I am assuming that it is a misspelling of lisure (``fun activity'').

\begin{exe}
    \ex{} \gll{}ge\~{}gu-isa geyle-kithi-sa ka-kas-lesure, kasa-pafay leyle-pah kas ifas.\\
                AGR\~{}ADJ-one game-INS-SG GDN-CONT-? PROX.DET-GEN name-NOM ? Ifas-PR\\
          \glt{}One game was used for continued leisure; this game was Ifas.
\end{exe}

\begin{exe}
    \ex{} \gll{}fasahri apal-kala pe\~{}gu-kata pelu-elah-pafay ka-lahkela giled-gula-tay,\\
                from.origin four-LOC AGR\~{}ADJ-ten person-PL-GEN GDN-make group.TFX-DU-ACC\\
          \glt{}Ten people from four locations split into two groups.
\end{exe}

This is grammatically incorrect, as case always precedes plurality: \textit{pelupafayelah}, not \textit{*peluelahpafay}.

\begin{exe}
    \ex{} \gll{}ka kada-giled-kala-sa fathas li-talo-kala kata-tay-sa.\\
                GDN each-group.TFX-LOC-SG see NEG-three-LOC ten-ACC-SG\\
          \glt{}Each guild saw not three, but ten.
\end{exe}

Ten of what?

\begin{exe}
    \ex{} \gll{}kada-giled-pafay-sa pasla-pah-sa esahla, ka pasla-pah-gula las laki, ka las suso-kala las kiksese-kala.\\
                each-guild-GEN-SG place-NOM-SG exist GDN place-NOM-DU COP big GDN COP forest-LOC COP field-LOC\\
          \glt{}Each guild's place existed, so two places were big, and forests were fields.
\end{exe}

This sentence was probably the hardest of them all. Using ``alu'' (``and'') may have conveyed the information better.

\begin{exe}
    \ex{} \gll{}kada-pasla-kala-gula ifigilepelu-pah-elah kufa su\~{}gu-talo sute-pah-elah lores gawralay-kala asila pasla-kala.\\
                each.every-place-LOC-DU in.group.person-NOM-PL coerce AGR\~{}ADJ-three stone-NOM-PL COP.3SG.PL ground-LOC DIST.DET place-LOC\\
          \glt{}In each and every place, a member would throw three stones onto that ground.
\end{exe}

\begin{exe}
    \ex{} \gll{}geyle-pah-sa ka-beretha-ta, ka ifigilepelu-pah-elah ka-lakad pa\~{}gu-iba pasla-kala-sa, ka ka-sakaki sute-tay-elah.\\
                game-NOM-SG INTEN-born-PST GDN in.group.person-NOM-PL INTEN-walk AGR\~{}ADJ-other place-LOC-SG GDN INTEN-search stone-ACC-PL\\
          \glt{}The game would then begin, and people would walk to places to look for stones.
\end{exe}

Interpreting "born" as "to begin, to start" here. Unless "born" is in the sense of "to start being a game regularly played", or "to become a common game".

For "to begin, start", Thiguka has \textit{kalirah}.

\section{New coins}
\begin{enumerate}
    \item suso n. forest, woodland, grove of trees \\ ← from Maacqu suusjong [suʃɔŋ] ́ “forest”
    \item ka conj. and, and so, and also, but also, also, and then (general discourse marker that keeps text flowing, connecting clauses) \\ ← shortened from Thiguka kalakala “in information” (information-LOC)
    \item kiksese n. field, meadow, open area, clearing \\ ← from Maacqu jaktiiksesee [jɐktiksɛsé] “meadow of kajs flowers” (lit. “that which is covered
in kajs flowers”)
    \item sakaki v. search for, look for, seek out \\ ← from Maacqu skakyy [skɐkɨ]́
\end{enumerate}

\section{In Lemuria's words}
\textbf{camelCaseCo version}
\begin{exe}
    \ex{} \gll{}kada-giled-pafay-sa pasla-pah-sa esahla, ka pasla-pah-gula las laki, ka las suso-kala las kiksese-kala.\\
                each-guild-GEN-SG place-NOM-SG exist GDN place-NOM-DU COP big GDN COP forest-LOC COP field-LOC\\
          \glt{}Each guild's place existed, so two places were big, and forests were fields.
\end{exe}

\textbf{Lemuria}
\begin{exe}
    \ex{} \gll{}Kada-giled-pafay-sa pasla-pah-sa esahla. Pasla-pah-gula pogusa      laki-tay alu suso-kala  gulaya      kiksese-kala.\\
                each-guild-GEN-SG   place-NOM-SG exist   place-NOM-DU   COP.3DU.PRS big-ACC  and forest-LOC COP.3SG.PRS field-LOC\\
          \glt{}
\end{exe}

\section{Final thoughts}
The translation was fairly challenging.
However, the grammatical errors are not camelCaseCo's fault; it is Lemuria's responsibility to comprehensively document his language.
Zethar had about a page of comments for Lemuria to address after Ambagame3 ended, pointing to the need for filling serious documentation gaps.

\end{document}
